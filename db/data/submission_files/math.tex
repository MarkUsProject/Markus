\documentclass{article}
\usepackage[utf8]{inputenc}

\title{Mathematical Expressions}

\author{An example from Overleaf}

\begin{document}

\maketitle

%-----------------------------------------------------------
The well known Pythagorean theorem $x^2 + y^2 = z^2$ was
proved to be invalid for other exponents.
Meaning the next equation has no integer solutions:

$$ x^n + y^n = z^n $$

%-----------------------------------------------------------
The well known Pythagorean theorem \(x^2 + y^2 = z^2\) was
proved to be invalid for other exponents.
Meaning the next equation has no integer solutions:

\[ x^n + y^n = z^n \]

%-----------------------------------------------------------
Here are some examples of simple usage of subscripts and superscripts:

\[ \int\limits_0^1 x^2 + y^2 \ dx \]

%-----------------------------------------------------------
\vspace{1cm}

Using superscript and subscripts in the same expression

\[ a_1^2 + a_2^2 = a_3^2 \]

%-----------------------------------------------------------
\vspace{1cm}

Longer subscripts and superscripts:

\[ x^{2 \alpha} - 1 = y_{ij} + y_{ij}  \]

%-----------------------------------------------------------
\vspace{1cm}

Nested subscripts and superscripts

\[ (a^n)^{r+s} = a^{nr+ns} \]

%-----------------------------------------------------------
\vspace{1cm}

Example of a mathematical equation with subscripts and superscripts

\[ \sum_{i=1}^{\infty} \frac{1}{n^s} = \prod_p \frac{1}{1 - p^{-s}} \]


%-----------------------------------------------------------
\vspace{1cm}

Squared root usage

\[ \sqrt[4]{4ac} = \sqrt{4ac}\sqrt{4ac} \]

\end{document}
